\usepackage[a4paper, left=3cm, right=2.5cm, top=2.5cm, bottom=3cm, headheight=20.6pt]{geometry}
\usepackage[utf8]{inputenc}

% Deutsche Sprache
\usepackage[ngerman]{babel}

\usepackage{graphicx}
\usepackage[parfill]{parskip} % paragraph new line
\usepackage{float}  % Bildposition

% Mathematik 
\usepackage{amsmath}
\usepackage{amssymb}
\numberwithin{equation}{section} % Numerierung mit Section

\usepackage[per-mode=symbol]{siunitx} % Einheiten
\sisetup{locale = DE}

% Kopfzeile und Fusszeile 
\usepackage{fancyhdr}

% Abkürzungsliste
% Abkürzungen im Text verwenden, damit sie in der Liste erscheinen.
\usepackage[printonlyused, withpage, smaller]{acronym}

\usepackage{nomencl}
\makenomenclature

% Generiert Mustertext Lorem ipsum
\usepackage{lipsum}

% linebraks within tables
\usepackage{makecell}

% Benutzerdefinierte Farben
\usepackage{soul}
\usepackage{xcolor}
\definecolor{mygreen}{rgb}{0,0.6,0}
\definecolor{mygray}{rgb}{0.98,0.98,0.98}
\definecolor{mymauve}{rgb}{0.58,0,0.82}
\definecolor{fhwn}{rgb}{0.09,0.19,0.45}

% Bibliothek 
\usepackage[backend=biber,style=ieee,citestyle=numeric-comp]{biblatex}
\addbibresource{Quellen.bib}
\setlength\bibitemsep{6pt}
\usepackage{csquotes} % Automatische "

% Überschrift Formatierung
\usepackage{titlesec}
\titleformat{\section}
{\color{fhwn}\normalfont\Large\bfseries}
{\color{fhwn}\thesection}{1em}{}

\titleformat{\subsection}
{\color{fhwn}\normalfont\bfseries}
{\color{fhwn}\thesubsection}{1em}{}

\titleformat{\subsubsection}
{\color{fhwn}\normalfont\bfseries}
{\color{fhwn}\thesubsubsection}{1em}{}

% Referenz / Beschriftungen
\usepackage{hyperref}
\usepackage[figure]{hypcap}

\makeatletter
\renewcommand{\fnum@figure}{Abb. \thefigure}
\renewcommand{\fnum@table}{Tab. \thetable}
\makeatother

\usepackage{subcaption}
\usepackage{caption}
\usepackage{listings}


\captionsetup{format=hang, strut=off}
\usepackage[figurewithin=section]{caption} % Figure count per section

% Änderung der Bezeichnung Listings -> Code
\renewcommand{\lstlistlistingname}{Programmcode}
\renewcommand{\lstlistingname}{Programmcode}

% Programmcode Darstellung
\lstset{
  backgroundcolor=\color{mygray},  % Hintergrundfarbe wählen - braucht \usepackage{color} oder \usepackage{xcolor};
  basicstyle=\scriptsize\ttfamily, % Schriftgröße 
  breakatwhitespace=false,         % sets if automatic breaks should only happen at whitespace
  breaklines=true,                 % neue Zeilen werden automatisch erstellt
  captionpos=b,                    % Beschreibung wird unten gesetzt b...bottom (und oben t...top)
  commentstyle=\color{mygreen},    % Farbe von erkannten Code Kommentaren 
  extendedchars=\true,             % Erlaubt nicht in ASCII enthaltene Zeichen zu benutzen, funktioniert nicht mit UTF8
  texcl=true,                      % Kommentare auch UTF-8
  firstnumber=1,                   % Beginn der Zeilenaufzählung mit 1, kann geändert werden im Fliesstext
  frame=tb,	                       % Rahmen um den Code 
  %frameround=tttt\thinlines,      % Rahmen bekommt mit t runde Ecken und mit f rechtwinkelige Ecken
  keepspaces=true,                 % 
  keywordstyle=\color{blue},       % Erkannte Schlüsselwörter werden blau markiert
  morekeywords={*,...},            % Schlüsselwörter benutzerdefieniert erweiterbar
  numbers=left,                    % Zahlenposition für die Aufzählung, mögliche Werte sind (none, left, right)
  numbersep=5pt,                   % Abstand der Zahlen zum Code
  numberstyle=\tiny\color{fhwn},   % Größe der Zahlen und die Farbe
  rulecolor=\color{fhwn},          % Farbe vom Rahmen
  showspaces=false,                % 
  showstringspaces=false,          % Strings sollen nicht unterstrichen werden
  showtabs=false,                  % tabs sollen nicht angezeigt werden
  stepnumber=1,                    % inkrement der Zeilennummer mit 1 --> jede Zeile wird numeriert 
  stringstyle=\color{mymauve},     % String wird violet markiert
  tabsize=2,                       % 
  framextopmargin=5pt,             % oberer Rahmenabstand zur ersten Zeile
  framexbottommargin=5pt,          % unterer Rahmenabstand zur letzten Zeile
  xleftmargin=.15\textwidth,       
  xrightmargin=.15\textwidth,
}

% Ermöglicht den Author und Titel im Text zu verwenden mit: OBSOLETE
% \runauthor \runtitle
%\makeatletter 
%\let\runauthor\@author
%\let\runtitle\@title
%\let\runsection\@section
%\makeatother

\newtoks\meinName

\newtoks\meinTitel


\newtoks\meineMartikelnummer

\newtoks\meinStudiengang

\newtoks\meinBetreuer

\newtoks\meinBegutachter



\newcommand{\MONTH}{%
  \ifcase\the\month
  \or Januar% 1
  \or Februar% 2
  \or März% 3
  \or April% 4
  \or Mai% 5
  \or Juni% 6
  \or Juli% 7
  \or August% 8
  \or September% 9
  \or Oktober% 10
  \or November% 11
  \or Dezember% 12
  \fi}

% Setzt die Section-Bezeichnung in den markright
\renewcommand{\sectionmark}[1]{\markright{#1}}
\renewcommand{\subsectionmark}[1]{}
